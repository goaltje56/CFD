\documentclass{CFD2017}
\usepackage{CFD2017}

%%%%%%%%%%%%%%%%%%%%%%%%%%%%%%%%%%%%%%%%%%%%%%%%%%%%%%%%%%%%%%%%%%%%%%%%%
%This template is created for complying with the author instructions for the
%12th International Conference on CFD in Oil & Gas, Metallurgical and Process Industries
%hosted by SINTEF/NTNU, Trondheim Norway
%May 30th - June 1st, 2017
%
%Questions/comments regarding bugs/errors in the template and associated files are welcome.
%Only limited support is given with respect to functionality, additional packages, or LaTeX in general.
%Please refer to e.g. https://www.latex-project.org/help/ for general LaTeX assistance and guidelines.
%
%
%Sverre G. Johnsen (sverre.g.johnsen@sintef.no)
%SINTEF Materials and Chemistry


%Required files:
%   ExampleFile.tex
%   ExampleFile.nls
%   CFD2017.bst
%   CFD2017.cls
%   CFD2017.sty
%   References.bib

%Example-specific files:
%   figure.eps
%   Table.tex
%%%%%%%%%%%%%%%%%%%%%%%%%%%%%%%%%%%%%%%%%%%%%%%%%%%%%%%%%%%%%%%%%%%%%%%%%




\title{The influence of a bluff body in a turbulent flow.}
\paperID{CFD 2018}
\author{Toos}{van Gool} %{forename}{surname}
\presenting  %the previous author is presenting the paper (name becomes underlined)
\address{University of Technology, Department of Mechanical Engineering, Eindhoven, NETHERLANDS}%affiliation of the previous author
\email{c.e.a.g.v.gool@student.tue.nl}%e-mail address of the previous author
\author{Oscar}{Meijer}
\address{University of Technology, Department of Mechanical Engineering, Eindhoven, NETHERLANDS}
\email{o.c.meijer@student.tue.nl}




\begin{document}
\maketitle  %create the title page
\headers   %create the page headers and footers


\abstract{
  This file is an example \LaTeX file for submission to CFD2018.  A
  limit of 15 pages applies (submitted file size < 10MB).
}
\keywords{
  CFD, Turbulence, Bluff body
}
\normalfont\normalsize



%\section{Nomenclature}
A complete list of symbols used, with dimensions, is required.
%the nomenclature needs to be entered into the file ExampleFile.nls

\printnomenclature[0.7cm]

\vskip .1em


\section{Introduction}
Industrial applications often make use of homogeneously mixed fluids or gases. PUT LITTLE EXAMPLE HERE.\\


The aim of this research is to find the influence of species density and turbulence intensity on fluid separation. A fluid or gas entering the system may include a total of $N$ species. Amongst the $N$ species, there may be groups of species with identical density. Computational Fluid Dynamics (CFD) is used to solve the system of equations. For this particular problem, the mass, momentum and energy equations are solved (NOG MEER??). The simulations are performed for fully developed turbulent and incompressible flows. The turbulence is modelled by means of a $\kappa-\epsilon$ model.\\
\newline
SOME PIECE ABOUT EXPECTATIONS??\\
\newline
The first part of this article is composed of the theory and the numerical solver used to attain the solution. Then CFD model is treated, after which the Geometry and boundary conditions applied to the problem are discussed. Subsequently the mesh convergence and executed simulations are explained. Finally, the results of the simulations are displayed with a discussion and conclusion.
\newpage



\section{Physics}
This chapter covers the theory behind the CFD model. Once the theory is explained, the discretization of the governing equations is treated. Additional to that the used solver is explained.

\subsection{Turbulent flow}
A flow in a channel can be characterized by the Reynolds number, a dimensionless number which represents the ratio of viscous and inertial forces in the flow. The Reynolds number can be calculated upon use of Equation \ref{Reynolds}\cite{Versteeg2007}. Here $\rho$ is the fluid density in $[kg\cdot m^{-1}]$, $u$ is the velocity in $[m\cdot s^{-1}]$, $D$ is the characteristic length in $[m]$ and $\mu$ is the fluids dynamic viscosity in $[kg\cdot m^{-1}\cdot s^{-1}]$

\begin{equation}
\label{Reynolds}
Re = \frac{\rho \boldsymbol{u} D}{\mu}
\end{equation}
The flow is considered turbulent if the Reynolds number exceeds a value of 2300. A turbulent flow is contemplated as chaotic, containing flow instabilities such as eddies. These eddies cause local fluctuations in the velocity field. Therefore, the normal steady convection and diffusion equations can not be used to solve turbulent flows. To solve the turbulent flow equations one needs to introduce general equation for unsteady convection and diffusion\cite{Versteeg2007}.\vspace{2mm}

\emph{Unsteady convection-diffusion}
\begin{equation}
\label{unsteadConvDiff}
\frac{\partial \rho \phi}{\partial t} + div(\rho\phi\mathbf u)= div(\Gamma grad \phi)+S_{\phi}
\end{equation}
Equation \ref{unsteadConvDiff} consists of four parts. The first part is the time dependent term, the second part represents the convection, the third part the diffusion and the last part of the equation is the sink and or source term. $\phi$ is the transport term, $\Gamma$ is the diffusion or conduction coefficient. The transport term changes according to the equation to be solved. \\
The conservation of mass, momentum and energy play a major role in a CFD code. All conservation equations are derived for the case of an incompressible flow. The conservation of mass can be extracted from Equation \ref{unsteadConvDiff}, and results in \cite{slides}:\vspace{2mm}

\emph{Mass balance}
\begin{equation}
\label{MassBal}
\frac{\partial \rho }{\partial t} + div(\rho\mathbf u)= 0
\end{equation}
For a two dimensional problem, the conservation of momentum needs to be solved in two directions, $x$ and $y$. The general form for the momentum conservation is treated as:\vspace{2mm}

\emph{Momentum balance}
\begin{equation}
\begin{split}
\label{MomBal}
&\frac{D( \rho u) }{D t} =-\frac{\partial P}{\partial x} + div(\mu \hspace{1mm} grad\hspace{1mm}u)+S_{Mx}\\ 
&\frac{D( \rho u) }{D t} =-\frac{\partial P}{\partial x} + div(\mu \hspace{1mm} grad\hspace{1mm}u)+S_{Mx}
\end{split}
\end{equation}\
The last conservation equation is the conservation of energy, and can be computed upon use of Equation \ref{EnerBal}.\vspace{2mm}

\emph{Energy balance}
\begin{equation}
\label{EnerBal}
\rho C_p\frac{D T }{D t} = div(k \hspace{1mm}grad\hspace{1mm}T) +S_i 
\end{equation}

Note that in equation \ref{EnerBal} the heat release by viscous dissipation is assumed to be zero. The viscous dissipation can be neglected if the velocity gradients within the domain are small. (DIT NOG EVEN GOED CHECKEN).\\
Since this study regards a turbulent flow, the mass and momentum equations have to be solved in a different manner. This is done by taking the Reynolds average approach. In the Reynolds average approach the mean velocity is determined and the turbulence is described by a fluctuation around the mean velocity. Therefore, components in the convection-diffusion and conservation equations are decomposed in a mean and a fluctuation variable. The decomposed variables are given in Equation \ref{decomposed}.\vspace{2mm}

\emph{Decomposed variables}
\begin{equation}
\label{decomposed}
u=U+u'; \hspace{4mm} v = V+v'; \hspace{4mm} w=W+w'; \hspace{4mm} p=P+p' 
\end{equation}
In here, $U$, $V$ and $W$ are mean the velocities in $x$, $y$ and $z$ direction respectively. $P$ is the average pressure. the apostrophe terms are the fluctuating components. The average term is given by Equation \ref{Uavg}.\vspace{2mm}

\emph{Average velocity}
\begin{equation}
\label{Uavg}
U = \frac{1}{\Delta t}\int_{0}^{\Delta t}u(t)dt
\end{equation}
The other average terms are calculated in the same manner. After computing the decomposed variables, they are substituted into the system of equations. The time averaged part in Equation \ref{Uavg} transforms the regular Navier-Stokes equations into time averaged Navier-Stokes equations, also called Reynolds averaged Navier-Stokes equations (RANS). By using RANS, a new definition is introduced called Reynolds stresses. The Reynolds stresses are present in all directions, and are calculated by Equation \ref{ReStress} \cite{slides}. \vspace{2mm}

\emph{Reynolds stresses}
\begin{equation}
\label{ReStress}
\tau_{ij}=-\rho\overline{u_i'u_j'}
\end{equation}
The $\kappa-\epsilon$ model is used to close the system of equations. The $\kappa-\epsilon$ is chosen above the Prandtl mixing length model and the Reynolds stress model and algebraic stress model due to its simple implementation and widely proven validation \cite{Versteeg2007}.


\subsubsection{$\kappa-\epsilon$ turbulence model}
The standard  $\kappa-\epsilon$ model introduces two extra transport equations to be solved. one for the turbulent kinetic energy, $\kappa$, and one for the viscous dissipation of turbulent kinetic energy, $\epsilon$.  $\kappa$ and $\epsilon$ are used to introduce a velocity scale, $\theta$ and large-scale turbulence length scale $\ell$. The velocity scale is calculated upon use of Equation \ref{velScale}, and the turbulence length-scale is calculated with use of Equation \ref{turbScale}.\vspace{2mm}

\emph{Velocity- and turbulence length-scale}
\begin{align}
&\theta = \kappa^{1/2} \label{velScale}\\
&\ell=\frac{\kappa^{3/2}}{\epsilon} \label{turbScale}
\end{align}
Equation \ref{turbScale} shows that one is able to use the small eddy variable $\epsilon$ to describe the large eddy scale, $\ell$. This is only permitted if, and only if, the rate at which large eddies extract energy from the mean flow matches the rate of transfer of energy across the energy spectrum to small, dissipating, eddies if the flow does not change rapidly. \cite{Versteeg2007}.\\

The turbulent viscosity, the eddy viscosity is introduced by Equation \ref{eddyvisc}, where $C_{\mu}$ is a dimensionless constant.\vspace{2mm}

\emph{Turbulent viscosity}
\begin{equation}
\label{eddyvisc}
\mu_t=C\rho \theta \ell=C_{\mu}\rho\frac{\kappa^2}{\epsilon}
\end{equation}

One is now able to describe the equations for both $\kappa$ and $\epsilon$, shown in Equation \ref{kappa} and \ref{eps} respectively.\vspace{2mm}

\emph{$\kappa$- and $\epsilon$-transport equation}

\begin{align}
&\frac{\partial \rho \kappa}{\partial t} + div(\rho \kappa \mathbf{U}) = div\bigg[\frac{\mu_t}{\sigma_{\kappa}}grad\hspace{1mm}\kappa\bigg] +2\mu_tS_{ij}\cdot S_{ij}-\rho\epsilon \label{kappa}\\
\begin{split}
&\frac{\partial \rho \epsilon}{\partial t} + div(\rho \epsilon \mathbf{U}) = div\bigg[\frac{\mu_t}{\sigma_{\epsilon}}grad\hspace{1mm}\epsilon\bigg] +\\
&C_{l\epsilon} \frac{\epsilon}{\kappa}2\mu_tS_{ij}\cdot S_{ij}-\rho C_{2\epsilon}\frac{\epsilon^2}{\kappa} \label{eps}
\end{split}
\end{align}

In both equations, the first term represents the rate of change in $\kappa$ and $\epsilon$ respectively. Within the first term $\mathbf{U}$ is the average velocity magnitude. The second and third term represent the transport driven by convection and diffusion for $\kappa$ and $\epsilon$. The fourth and fifth term are the rate of production and rate of destruction of $\kappa$ and $\epsilon$.\\
The equations use five dimensionless constants, the standard $\kappa-\epsilon$ model uses values that have been determined through comprehensive data fitting for a wide range of turbulent flows \cite{Versteeg2007}:\vspace{2mm}

\emph{Dimensionless constants:}
\begin{equation*}
\label{dimConst}
C_{\mu}=0.09 \hspace{4mm} \sigma_{\kappa}=1.00\hspace{4mm}\sigma_{\epsilon}=1.30\hspace{4mm}C_{l\epsilon}=1.44\hspace{4mm}C_{2\epsilon}=1.92
\end{equation*}


\subsubsection{discretization of governing equations}
To solve the turbulence model, all of the partial differential equations (PDE's) need to be solved. Solving a PDE takes a few steps. First of all let us consider a two dimensional grid consisting of several grid cells. To find the value for $\phi$ from Equation \ref{unsteadConvDiff} at a centre point P of the grid cell, one needs the values of the neighbouring cells. In CFD these values are called north(N), west(W), south(S) and east(E). Thereafter, Equation \ref{unsteadConvDiff} is integrated over the control volume, yielding the results shown in Equation \ref{intergrated}.\vspace{2mm}
DEZE FUNCTIE NOG EENS NAKIJKEN WANT DIE IS MOEILIJK\\
\emph{Intergrated diffusion convection equation}
\begin{equation}
\begin{split}
\label{intergrated}
&\frac{(\rho_p \rho_p^0)\phi}{\Delta t}\Delta V +[(\rho u A \phi)_e -(\rho u A \phi)_w]+[(\rho u A \phi)_n -\\
&(\rho u A \phi)_s]=\bigg[\bigg(\Gamma A \frac{\partial \phi}{\partial x}\bigg)_e -\bigg(\Gamma A \frac{\partial \phi}{\partial x}\bigg)_w\bigg] +\\
&\bigg[\bigg(\Gamma A \frac{\partial \phi}{\partial x}\bigg)_n -\bigg(\Gamma A \frac{\partial \phi}{\partial x}\bigg)_s\bigg] + S_u + S_p\phi_p^0
\end{split}
\end{equation}
Now let us introduce variables $F$ and $D$ to represent the convective mass flux per unit area and the diffusive conductance at the cell faces \cite{Versteeg2007}.\vspace{2mm}

\begin{align}
&F = \rho u\\
&D=\frac{\Gamma}{\partial x}
\end{align}
Substitution variables $F$ and $D$ into Equation \ref{intergrated} forms Equation \ref{Substituted} after some rearranging \cite{Versteeg2007}. \\OOK DEZE FORMULE NOG CHECKEN!!!

\begin{equation}
\label{Substituted}
\frac{\rho_P^0 \Delta V}{\Delta t}\phi_P=a_W\phi_W+a_E\phi_E+a_S\phi_S+a_N\phi_N +\Delta F
\end{equation}
The coefficients of $a$ are dependent on the differencing scheme. The discretization scheme has some properties. First property is the conservativeness which means the flux $\phi$ leaving a control volume has to be equal to the flux $\phi$ entering the control volume through the same face. By using a hybrid differencing scheme, the conservativeness is automatically satisfied. The second property is the boundedness. The boundedness describes the value of $\phi$ in case there is no source term. In that case, the nodal value of $\phi$ must be equal to the value of $\phi$ at the boundary. Besides the restricted value of $\phi$, the coefficients $a$ of the discretized equation must have equal signs. The last important property is the transportiveness. This requires that the transportiveness changes according to the magnitude of the P\`{e}clet number ($Pe=F/D$). Hence, if Pe is zero, $\phi$ equally diffuses in all directions. The hybrid differencing scheme satisfies the three properties. However, this comes at the price of only first order accuracy \cite{Versteeg2007}. The obtained values for $F$ and $D$ determine the values of the $a$ coefficients. They can be determined as shown in Table \ref{tab:coefsa}.
\InsTab{coefficients}\\
The discretized equation can now be solved using the transient SIMPLE algorithm. SIMPLE stands for Semi-Implicit Method for Pressure-Linked Equations. In essential the algorithm is a guess and correct procedure for pressure calculations in a staggered grid which is used \cite{Versteeg2007}. The transient SIMPLE algortithm is used due to the unsteady convection-diffusion equations used to describe the problem. The difference with the regular SIMPLE algorithm is the time $t$, discretized in time steps, $\Delta t$ for which the equations are solved. The transient SIMPLE algortithm works in the following manner. First the variables $u$, $v$, $p$ and $\phi$ are initialized and a time step $\Delta t$ is defined. The time itiration starts at $t_0$ and continues untill $t>t_{max}$. VERDERE UITLEG NODIG, EVEN GOED BEKIJKEN HOE DIE SOLVER WERKT

\section{CFD problem}
This chapter will describe the entire CFD model used during simulations. First of all the geometry will be introduces, after which the applied boundary conditions are explained.

\subsubsection{Geometry}
Figure \ref{fig:Geometry} shows the geometry used during the research. The rectangular grid with dimensions length, $L$ and width $D$, contains a bluff-body with height and width $h$ and $b$ respectively.

\InsFig{FlowGeometry}{Schematic diagram of geometry.}{Geometry}

The size of the bluff body is changer during the simulations for the purpose of finding its influence on the turbulence intensity. The size of the bluff-body will from now on be described with respect to the width of the channel. This can be acchieved upon use of $h/D$, wich describes the dimensionless size.\\



\subsubsection{General boundary conditions}
To compose a realistic problem one needs to define boundary conditions. Since this study focussus on tubulent flows only, the applied boundary conditions are solely turbulent as well. This section will cover the boundary conditions applied to the general domain. The next Section \ref{bluffBC} will cover the boundary conditions applied to the bluff body.  will  First of all A fully developed turbulent profile enters the geometry at the left side. This boundary condition is applied to reduce computational time. Without the fully developed inlet profile, the flow needs some time to adapt to a fully developed profile, which demands computational power. The inlet boundary condition can be applied with help of Equation \ref{inletBC}, called the power law velocity profile. \vspace{2mm}

\emph{Power law velocity profile}
\begin{equation}
\label{inletBC}
u(y) = \begin{cases} U_{IN}\cdot \big[\frac{y}{D/2}\big]^{1/n}, & \mbox{if } y\leq\frac{D}{2} \\ U_{IN}\cdot \big[2-\frac{y}{D/2}\big]^{1/n}, & \mbox{if } y>\frac{D}{2}\end{cases}
\end{equation}
The equation is constructed of a lower and upper half since the geometry has its origin in the lower left corner. In the equation $D$ is defined as the maximum height in the channel, $U_{IN}$ is the maximum inlet velocity. The height of the profile can be altered by a change in $U_{IN}$. The power $n$ is chosen to be 7, which is a value suited for a wide range of turbulent flows \cite{Morrison}. Figure \ref{fig:Inlet} shows the analytical inlet, as presented in Equation \ref{inletBC}, the inlet profile as extracted from the simulation and the profile at the exit of the geometry as extracted from the simulation. The entire domain is initialized to a temperature of 273$K$.

\InsFig{InletBC}{Schematic diagram of geometry.}{Inlet}

One can clearly see the agreement of the analytical expression for the inlet profile with the numerical result at the inlet. The outlet however, shows a sligth deviation from the inlet condition. This is due to the fact that the analytical expression is an approximation of the turbulent profile.\\
To obtain accurate results one has to define values for the turbulent kinetic energy and dissipation rate at the inlet of the geometry. The inlet values can be calculated upon use of Equation \ref{kepsBC}.

\begin{equation}
\label{kepsBC}
\kappa_{IN}=\frac{2}{3}(U_{IN}T_i)^2,\hspace{4mm}\epsilon_{IN}=C_{\mu}^0.25\frac{\kappa_{IN}^{1/3}}{0.035 D}
\end{equation}
In these equations $T_i$ is the turbulent intensity, set to a value of 4\%. The value for $T_i$ for the inlet boundary is in general taken between 1\% and 6\% \cite{Versteeg2007}.\\
At both top and bottom walls of the geometry the axial velocity is set to $u=0[m\cdot s^{-1}]$. Since the flow is turbulent not only the value at the boundary itself should be restricted, but also the so called near wall region. These regions need wall functions to describe the flow near the walls. Closest to the walls there is a viscous sub-layer. Within this layer the eddies decrease in size and dissipate their energy due to viscous forces dominating in this area. When one looks more towards the centre of the geometry a log-law layer appears. In order to implement such conditions one needs to introduce $y^+$ and $u^+$, as seen in Equation \ref{yplusuplus} \cite{Versteeg2007}.

\begin{equation}
\label{yplusuplus}
y^+=\frac{\rho u_{\tau}y}{\mu},\hspace{4mm}u^+=\frac{U}{ u_{\tau}}
\end{equation}
In this equation, $ u_{\tau}$ represents the friction velocity and $U$ is the average mean flow velocity. The friction velocity is calculated with Equation \ref{fricvel}
\begin{equation}
\label{fricvel}
u_{\tau}=\big(\frac{\tau_w}{\rho}\big)^{1/2}
\end{equation}
$\tau_w$ is the shear stress in the boundary layer in $[Pa]$. In the viscous sub-layer, which is defined in the layer where $y^+\leq 11.63$, the value of $u^+$ is equal to the value of $y^+$. The shear stress in this region can be evaluated by Equation \ref{subtau}

\begin{equation}
\label{subtau}
\tau_w = \mu \frac{\partial u}{\partial y}
\end{equation}
If the value of the dimensionless height $y^+$ grows beyond $11.63$ one enters the log-law regime. Within this layer $u^+$ is calculated with Equation\ref{ulog}, where the values for $\kappa$ and $E$ are $0.4187$ and $9.793$ respectively.
\begin{equation}
\label{ulog}
u^+=\frac{1}{\kappa}ln(Ey^+)
\end{equation}
The shear stress in the log-law area, used to calculate $y^+$, is determined upon us of Equation \ref{taulog}
\begin{equation}
\label{taulog}
\tau_w = \frac{\rho C_{\mu}^{0.25}\kappa^{1/2}\partial u}{u^+}
\end{equation}


 At the exit of the geometry the gradient of velocity and temperature in both $u$ and $y$ directions set to zero. This is done in order to comply with the mass balance equation since otherwise mass may disappear through the exit.\\

\subsubsection{Bluff-body boundary conditions} \label{bluffBC}
The walls of the bluff body are treated in the same manner as explained in the previous section. Additional to that, the grid cells inside the bluff body are set to zero values by prescribing $S_p=-10^{30}$. The temperature of the bluff body is prescribed at the walls, and can be changed. MOETEN WE OOK DIE KAPPA EN EPSILON UITLEGGEN BIJ DEZE BOUNDARY CONDITIONS?????

 
\section{Results}
The results of using the \LaTeX template is a great looking paper.
In Figures \ref{fig:Geometry} and \ref{fig:Geometry} it can be seen how figures are easily included.
In Table \ref{tab:label} it is seen how we can include a table.
The table is constructed in the file table.tex, where also the table caption and label are defined.

\InsFig{convergence_x10_y20}{Turbulent inlet profile}{conv_1}[H]
\InsFig{convergence_x20_y40}{Turbulent inlet profile}{conv_2}[H]
\InsFig{convergence_x40_y80}{Turbulent inlet profile}{conv_3}[H]
\InsFig{convergence_x80_y160}{Turbulent inlet profile}{conv_4}[H]
\InsFig{Re_finest}{Reynolds number}{Re}
\newpage
\InsTab{Table}

\section{Conclusion}
The conclusions are:
\begin{enumerate}
  \item Trondheim is a nice city.
  \item CFD is great fun, and useful too.
\end{enumerate}


%% %---------------------------
%% %BIBLIOGRAPHY
%% %---------------------------
%The bibliography is created using BiBTeX
\bibliographystyle{CFD2017}
\bibliography{References}


\newpage
\section{Appendix A}
Give any additional information here.

\end{document} 